\newpage
\section{Ausblick}
\label{ausblick}
Ein wichtiger Bestandteil der Betrachtung, die hier wegen fehlender Zeit ausgelassen wurde, ist der Umgang des Agenten mit dynamischen Hindernissen. Einkaufsroboter müssen dazu in der Lage sein mit dem Verhalten von Menschen umzugehen. Diese weisen in einem realen Umfeld unterschiedliche Muster auf. Teilweise würden sie dem Roboter ausweichen, andere stellen sich ihm vielleicht aktiv in den Weg oder bleiben an Ort und Stelle stehen. Herauszufinden wie der Roboter dieser vorm von Hindernissen effizient ausweicht, während er weiterhin versucht, alle Artikel der Einkaufsliste zu kaufen, könnte Bestandteil einer weiterführenden Arbeit sein. 
\\
Ebenso ist die Implementierung eines Vergleichsmodell von Interesse. Dieses könnte beispielsweise ein regelbasiertes System sein, welches gleichbleibend mit Hindernissen umgeht. Für die Betrachtung der verschiedenen Systeme würde man diese in getrennten Umgebungen, die über dieselben Hindernisse und zu kaufende Artikel verfügen, gegeneinander antreten lassen. Durch die Zufälligkeit der Umgebung kann nicht gewährleisten werden, dass die simulierten Menschen nach Kontakt mit dem Roboter gleich agieren. Dennoch würde die Nutzung eines Vergleichsmodells Anhaltspunkte über die Effizienz des KI-Agenten geben. Durch diese Form der Betrachtung könnte man einfacher die Stärken und Schwächen der jeweiligen Ansätze aufschlüsseln. Hierfür sollten Gesichtspunkte wie Dauer des Einkaufs, Anzahl an Kollisionen und Verärgerung der simulierten Menschen analysiert werden. Wenn man den Agenten auf Basis dieser Gesichtspunkte optimiert, wäre das Ergebnis, ein Roboter, der so wenig wie möglich Fahrtweg zurücklegt, während er die anderen Einkäufer minimal stört und die Dauer des Einkaufs gering hält. 
\\
Zusätzlich sollten Vereinfachungen, die in dieser Implantierung vorgenommen wurden, sukzessive abgebaut werden. Des Weiteren könnten noch komplexere Regalstrukturen, wie man sie aus dem Edeka oder Rewe kennt, hinzugefügt werden. Desto weniger Simplifizierungen verwendet werden, umso aussagekräftiger sind die Ergebnisse.
