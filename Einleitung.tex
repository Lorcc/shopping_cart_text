\newpage
\section{Einleitung}
\label{einleitung}
Während die Übernahme des Einkaufes für manche eine reine Zeitersparnis wäre, löst es für körperlich eingeschränkte Personen ein bedeutendes Problem. Das Transportieren eines schweren Wocheneinkaufs stellt für ältere und eingeschränkte Personen, die häufig keinen Zugang zu Autos haben eine Herausforderung dar. Die Übernahme des Einkaufes könnte auch diesen Aspekt vereinfachen, da der Transport des Einkaufes durch Drittunternehmen wie beispielweise Flaschenpost ermöglicht wird, ohne ein eingeschränktes Angebot in Kauf nehmen zu müssen.
\\
Jedoch ist für viele der Einkauf ein üblicher Bestandteil des Lebens, weswegen hier kein Raum geschaffen werden kann, in dem Mensch und automatisierte Roboter voneinander getrennt agieren können. Ebenso ist eine Trennung nicht im Interesse der Supermärkte, da dies zu weniger impulsiven Käufen und somit zu weniger Gewinnen führen würde. Deswegen ist ein Kontakt des Einkaufsroboters mit Menschen unausweichlich. 
\\
Andere Universitäten haben hier ebenso ein großes Potential erkannt. Mit dem Projekt I-RobEka verfolgt die TU Chemnitz einen Ansatz zur Implementierung eines voll umfänglichen Einkaufsroboters. Ein anderer Zweig in dem Forschungsbereich sind unterstützende Einkaufsroboter, deren Aufgaben weniger weitreichend sind. Meist befassen sich diese mit dem Tragen des Einkaufs oder sie geben Einkaufenden Wegbeschreibungen, damit diese die Produkte schneller finden können. Mehr zu dem derzeitigen Forschungsstand im Bereich Reinforcement Lerning und Einkaufsroboter berichte ich im Kapitel …. 
\\
Das Ziel dieser Arbeit ist herauszufinden, ob mithilfe von Reinforcement Learning ein lernender Agent geschaffen werden kann, der dazu in der Lage ist in einem simulierten Supermarkt einen Einkauf zu übernehmen. Hierfür werden unterschiedliche Szenarien getestet und analysiert. Gesichtspunkte sind Realisierbarkeit in der Realität sowie Dauer des Einkaufs und die Häufigkeit von Kollisionen. 
\\
Einkaufsroboter müssen dazu in der Lage sein, mit den sich verändernden Verhältnissen innerhalb des Supermarktes umzugehen. Dies gilt vor allem im Bezug auf die Wegplanung. Während das grundsätzliche Kartografieren des Supermarktes simpel ist, kann dies nicht über die Position der tausenden Artikel gesagt werden. Die Implementierung einer solchen Datenbank führt zu einem erheblichen Aufwand und zu einer bestehenbleibenden Wartungsaufgabe. Dennoch wird für diese Projekt angenommen, dass die Position der verschiedenen Artikel innerhalb des Supermarkts bekannt ist. Die Information zur Position der Artikel wird daraufhin genutzt, um den kürzesten Weg für den gesamten Einkauf zu berechnen. Dieser Weg kann wahrscheinlich nicht so genutzt werden, da nicht nur relativ statische Hindernisse wie Einkaufswägen, sondern auch Menschen den Weg versperren können. Somit muss der Roboter entscheiden, wie er den Hindernissen effizient ausweicht, während er weiterhin versucht alle Artikel der Einkaufsliste zu kaufen. Die Datenbank mit den Positionen der Artikel ist nur eine von vielen Annahmen zur Umsetzung dieses Projektes. Mehr zu den Annahmen kann im Kapitel … gefunden werden.
\\
Die Bewegungssteuerung des Roboters wird durch Reinforcement Learning trainiert. Dafür muss der Agent lernen, den berechneten Weg so gut wie möglich zu folgen. Das Einsammeln der Artikel wird simplifiziert. Die Wahrnehmung der Umgebung erfolgt durch am Agenten angebrachte simulierte Sensoriken. Weiterhin erhält der Roboter Informationen über sich selbst wie bspw. die eigene Geschwindigkeit.
