\newpage
\section{Getroffene Annahmen für die Implementierung des Projekts}
\label{annahmen}
Für die Implementierungen dieses Projekts wurden diverse Annahmen getroffen, um die Modellierung dieser Aufgabe zu vereinfachen und überhaupt erst zu ermöglichen. Die Annahmen können hierbei in zwei Bereiche unterteilt werden. Zum einen betreffen diese die Strukturierung des Supermarktes. Zum anderen die Aktionen und die Sensorik des Agenten.
\\
Um die Simulation des Supermarktes in der Wirklichkeit zu verankern, habe ich mir mehrere Supermärkte angeschaut und den hier ansässigen Edeka ausgemessen. Dabei habe ich einige grundsätzliche Informationen zusammentragen können.
\\
\begin{itemize}
	\item Die Abstände zwischen den Regalen sind mindestens so breit, dass zwei Einkaufswägen aneinander vorbeipassen, hierbei gibt es einige wenige Ausnahmen in Verbindung mit Aktionsaufstellern, diese werden aber außenvorgelassen.
	\item Strukturell beginnen die meisten Supermärkte mit frischen Produkten wie Obst, Gemüse, Brot, usw. Danach folgen häufig Kühlbereiche mit Fertigprodukten. Daraufhin länger haltbare Nahrungsmittel wie Süßigkeiten, Dosen, orientalische Produkte, usw. Zum Schluss folgen oft Getränke.
	\item Es gibt keine Sackgassen. Der Weg in das Abteil ist nie der einzige weg hinaus.
	\item Meist wird versucht ein möglichst hohes Maß an Abdeckung mit Regalen zu erreichen.
\end{itemize}
\noindent
\\
Auch wenn dies im Realen nicht immer gilt wird für die Simulation angenommen, dass der Grundschnitt des Supermarktes sowie die Aufteilungen der Abteile quadratisch oder rechteckig sind. Weitere Daten, die ich durch die Ausmessung erhielt, habe ich in die Modellierung des Supermarktes einfließen lassen. Dazu gehört beispielsweise die Höhe der Regale, die Höhe des höchsten Regalfaches, die Abstände zwischen den Fächern und weitere Elemente.
\\
Des Weiteren wurde für die Berechnung der Wegplanung angenommen, dass der Supermarkt mit der derzeitigen Regalstruktur kartografiert wurde. Zu dieser Karte gehören nur statische Elemente, die sich über die Zeit hinweg nicht ändern. Hindernisse wie Einkaufswägen und Pappboxen mit neuen Artikeln werden nicht miteinbezogen. Sollte sich die Regalanordnung ändern, muss auch die Karte erneuert werden. Ein Ansatz mit einer dynamischen Karte, in der Hindernisse während des Trainingsprozesses vom Agenten hinzugefügt oder gelöscht werden, wird in dieser Arbeit nicht betrachtet. Grundlage hierfür ist, einerseits die Rechenersparnis hinsichtlich der festgelegten Wegplanung. Andererseits ist die Betrachtung der Reaktion des Agenten auf unbekannte Hindernisse ein wichtiger Bestandteil der Arbeit. Das Kartografieren des Supermarktes ist simpel und realistisch umsetzbar. Jedoch gilt dies nicht für die Position der Artikel innerhalb des Supermarktes. Hierfür wird angenommen, dass eine Datenbank besteht, die diese Position eingespeichert hat und innerhalb der Karte darstellen kann. Diese Annahme ist eher unrealistisch, da die Datenbank ständig angepasst werden müsste, sobald die Produkte innerhalb der Regale umsortiert werden, oder neue Artikel hinzukommen. Des Weiteren ist dies besonders fehleranfällig, da der Agent davon ausgeht, dass der ihm übergebene Weg, der Richtige ist. Falls aber die Datenbank nach dem Umräumen nicht vollständig angepasst wurde, würde der Agent ein falsches Produkt einsammeln. Dennoch wurde diese Annahme getroffen, um in der vorgegebenen Zeit dieses Projekt umsetzen zu können. Die Alternative zu diesem Ansatz, wäre die Entwicklung einer Künstlichen Intelligenz, welche die Produkte mithilfe einer Kamera erkennen kann. Ohne die potenziellen Probleme dieser Vorgehensweise zu beschreiben, wäre hierfür zunächst ein Datensatz mit tausenden Bildern der jeweiligen Produkte aus verschiedenen Perspektiven notwendig. Die Implementierung einer solchen KI als Instanz vor der eigentlichen Untersuchung übersteigt den Rahmen einer Masterarbeit. 
\\
Zusätzlich wird das Einsammeln der Artikel simplifiziert und dabei wird getestet, ob der Roboter zur berechneten Position fährt. Das physische Einsammeln von Objekten verschiedener Größe, während der Roboter nicht unbedingt perfekt ausgerichtet ist, ist hochkomplex, weswegen diese Form der Vereinfachung gewählt wurde.
\\
Dies sind die hauptsächlichen Annahmen, die getroffen wurden. Des Weiteren gibt es zusätzliche Anpassungen, um die Berechnung der Simulation zu optimieren. Für die Kollisionsberechnung der Regale und der Kasse wurde eine große Hitbox benutzt. Für mehr Realismus hätte man hier die genau Form der Regale durch Hitboxen nachbauen können. 
\\
Des Weiteren wurden keine zufälligen Fehler zu den Sensordaten und auch der Steuerung hinzugefügt. Alle Berechnungen wurden genau übergeben. Dies lag daran, dass viel Zeit aufgewendet
