\documentclass[doktyp=marbeit, xcolor=table]{TUBAFarbeiten}
\usepackage[onehalfspacing]{setspace}
\usepackage[center]{caption}
% Bar chart drawing library 
\usepackage{pgfplots}
\pgfplotsset{compat=newest}
\usetikzlibrary{decorations.pathmorphing}
\usepackage{graphicx}
\usepackage{subfig}
\usepackage{picture}
\usepackage[hyphens]{url}
\usepackage[hidelinks]{hyperref}
\usepackage{xargs}
\usepackage{selinput}
\SelectInputMappings{adieresis={ä}, germandbls={ß}, Euro={€}}
\usepackage[T1]{fontenc}
\usepackage{csquotes}
\usepackage{multirow}
\usepackage{mathtools}
\usepackage{multicol}
\usepackage{amsmath}
\usepackage{listings}
\usepackage[backend=biber, sortlocale=de_DE_phonebook, style=numeric, sorting=none]{biblatex}


\usepackage{lmodern}
\usepackage{bchart}
\usepackage{booktabs}
\usepackage{makecell}

\definecolor{commentgreen}{RGB}{2,112,10}
\definecolor{eminence}{RGB}{108,48,130}
\definecolor{weborange}{RGB}{255,165,0}
\definecolor{frenchplum}{RGB}{129,20,83}

\renewcommand{\lstlistingname}{Liste}

\lstset {
	language=C++,
	frame=tb,
	tabsize=4,
	showstringspaces=false,
	numbers=left,
	%upquote=true,
	basicstyle=\small\ttfamily, % basic font setting
	emph={int,char,double,float,unsigned,void,bool},
	emphstyle={\color{blue}},
	escapechar=\&,
	% keyword highlighting
	classoffset=1, % starting new class
	otherkeywords={>,<,.,;,-,!,=,~},
	morekeywords={>,<,.,;,-,!,=,~},
	keywordstyle=\color{weborange},
	classoffset=0,
	breaklines=true,
	breakatwhitespace=true,
}

\setcounter{biburlnumpenalty}{100}
\setcounter{biburlucpenalty}{100}
\setcounter{biburllcpenalty}{100}

\addbibresource{master.bib}

\TUBAFTitel[Konzeption und Implementierung eines Einkaufroboters in einem simulierten Supermarkt auf Basis von Reinforcement Learning] {
	Konzeption und Implementierung eines Einkaufroboters in einem simulierten Supermarkt auf Basis von Reinforcement Learning}
%\titleformat{\section}{\fontseries semibold}
\TUBAFBetreuer{Prof. Dr. Bernhard Jung}
\TUBAFKorrektor{Dr. Stefan Reitmann}
\TUBAFAutor[M. Hoffmann] {Marc Hoffmann}
\TUBAFFakultaet{Fakultät für Mathematik und Informatik}
\TUBAFInstitut {Institut für Informatik}
\TUBAFLehrstuhl{Professur für Virtuelle Realität und Multimedia}
\TUBAFStudiengang{Master Angewandte Informatik}
\TUBAFMatrikel{61\,090}
\TUBAFDatum[2024-03−14]{14. März 2024}
\pdfmapfile{=bfu.map}


\begin{document}
	\maketitle
	\TUBAFErklaerungsseite
	\tableofcontents
	\listoffigures
	\listoftables
	\newpage
\section{Einleitung}
\label{einleitung}
Während die Übernahme des Einkaufes für manche eine reine Zeitersparnis wäre, löst es für körperlich eingeschränkte Personen ein bedeutendes Problem. Das Transportieren eines schweren Wocheneinkaufs stellt für ältere und eingeschränkte Personen, die häufig keinen Zugang zu Autos haben eine Herausforderung dar. Die Übernahme des Einkaufes könnte auch diesen Aspekt vereinfachen, da der Transport des Einkaufes durch Drittunternehmen wie beispielweise Flaschenpost ermöglicht wird, ohne ein eingeschränktes Angebot in Kauf nehmen zu müssen.
\\
Jedoch ist für viele der Einkauf ein üblicher Bestandteil des Lebens, weswegen hier kein Raum geschaffen werden kann, in dem Mensch und automatisierte Roboter voneinander getrennt agieren können. Ebenso ist eine Trennung nicht im Interesse der Supermärkte, da dies zu weniger impulsiven Käufen und somit zu weniger Gewinnen führen würde. Deswegen ist ein Kontakt des Einkaufsroboters mit Menschen unausweichlich. 
\\
Andere Universitäten haben hier ebenso ein großes Potential erkannt. Mit dem Projekt I-RobEka verfolgt die TU Chemnitz einen Ansatz zur Implementierung eines voll umfänglichen Einkaufsroboters. Ein anderer Zweig in dem Forschungsbereich sind unterstützende Einkaufsroboter, deren Aufgaben weniger weitreichend sind. Meist befassen sich diese mit dem Tragen des Einkaufs oder sie geben Einkaufenden Wegbeschreibungen, damit diese die Produkte schneller finden können. Mehr zu dem derzeitigen Forschungsstand im Bereich Reinforcement Learning und Robotik im Einkauf berichte ich im Kapitel \ref{forschung}. 
\\
Das Ziel dieser Arbeit ist herauszufinden, ob mithilfe von Reinforcement Learning ein lernender Agent geschaffen werden kann, der dazu in der Lage ist in einem simulierten Supermarkt einen Einkauf zu übernehmen. Hierfür werden unterschiedliche Szenarien getestet und analysiert. Gesichtspunkte sind Realisierbarkeit in der Realität sowie Dauer des Einkaufs und die Fähigkeit Hindernissen auszuweichen. 
\\
Einkaufsroboter müssen dazu in der Lage sein, mit den sich verändernden Verhältnissen innerhalb des Supermarktes umzugehen. Dies gilt vor allem im Bezug auf die Wegplanung. Während das grundsätzliche Kartografieren des Supermarktes simpel ist, kann dies nicht über die Position der tausenden Artikel gesagt werden. Die Implementierung einer solchen Datenbank führt zu einem erheblichen Aufwand und zu einer bestehen bleibenden Wartungsaufgabe. Dennoch wird für diese Projekt angenommen, dass die Position der verschiedenen Artikel innerhalb des Supermarkts bekannt ist. Die Information zur Position der Artikel wird daraufhin genutzt, um den kürzesten Weg für den gesamten Einkauf zu berechnen. Dieser Weg kann wahrscheinlich nicht so genutzt werden, da nicht nur relativ statische Hindernisse wie Einkaufswägen, sondern auch Menschen den Weg versperren können. Somit muss der Roboter entscheiden, wie er den Hindernissen effizient ausweicht, während er weiterhin versucht alle Artikel der Einkaufsliste zu kaufen. Die Datenbank mit den Positionen der Artikel ist nur eine von vielen Annahmen zur Umsetzung dieses Projektes. Mehr zu den Annahmen kann im Kapitel \ref{generierung_sim} gefunden werden.
\\
Die Bewegungssteuerung des Roboters wird durch Reinforcement Learning trainiert. Dafür muss der Agent lernen, den berechneten Weg so gut wie möglich zu folgen. Das Einsammeln der Artikel wird simplifiziert. Die Wahrnehmung der Umgebung erfolgt durch am Agenten angebrachte simulierte Sensorik auf die im Kapitel \ref{struktur} eingegangen wird. Weiterhin erhält der Roboter Informationen über sich selbst wie bspw. die eigene Geschwindigkeit. 

	\newpage
\section{Forschungsstand in den jeweiligen Bereichen}
\subsection{Reinforcement Learning}
\subsection{Einkaufsroboter}
	\newpage
\section{ML-Agents und Unity}
\label{mlagents}
Für die Umsetzung des RL-Agenten wird das Open Source Projekt ML-Agents und die Spiel-Engine Unity verwendet. Unity wird primär zur Entwicklung von 3D Umgebungen für Spiele, Simulationen, Filme und vieles mehr.\cite{unity} In Unity erstellt man häufig Objekte, die daraufhin mit Komponenten ausgestattet werden. Zu diesen Komponenten gehören beispielsweise Materialien, Collision Shapes (dt.: Kollisionsformen) aber auch Skripte, die das Verhalten des Objekts bestimmen. Somit wird ein Objekt mit vielen Funktionalitäten ausgestattet. Den Objekten können noch weitere sogenannte Child Objects (dt.: Kind Objekte) angehangen werden. Somit baut sich ein Objekt wie ein Baum mit seinen Verästlungen auf. Dieser Ansatz ist für Spiele-Engins nicht unüblich und wird beispielsweise auch in Godot verwendet. Warum also ausgerechnet Unity nutzen? 
\\
Die Engine dient als Basis für dieses Projekt, da hierfür das vorher angesprochene Machine Learning Agents Toolkit erstellt wurde. Das ML-Agents Toolkit ist ein Open-Source-Projekt, bei dem jeder der möchte, zur Weiterentwicklung beitragen kann. Es ermöglicht eine einfache Implementierung von Umgebungen zum Trainieren von intelligenten Agenten.\cite{ml_agents} Dafür liefert ML-Agents Grundbausteine für die Implementierung von Reinforcement Learning in Unity. Dazu gehört eine Vielzahl an Klassen, Komponenten und Skripts. 
\\
Zusätzlich lassen sich die erworbenen Daten einfach via TensorBoard visualisieren.\cite{tensorboard} TensorBoard kann verschiedene Metriken des trainierten Agenten in Grafiken wiedergeben und somit den aktuellen Stand des Trainingsprozesses darstellen. Dazu gehört der Reward zu einem gegebenen Zeitpunkt, die durchschnittliche Dauer der Episoden, die Lernrate und vieles mehr. Dafür muss folgender Befehl in der Command Promp an der Position des Projekts ausgeführt werden: 
\\
\begin{lstlisting}[language=bash,numbers=none]
	C:\PfadzumProjekt> venv\Scripts\activate
	(venv) C:\PfadzumProjekt> tensorboard --logdir results
\end{lstlisting}
\noindent
\\
Der erste Befehlt starte die virtuelle Umgebung (eng.: virtual environment) und der zweite Befehl öffnet die gespeicherten Daten in dem <results> Ordner. In diesem werden die Ergebnisse automatisch während des Trainingsprozesses gespeichert.
\\
Des Weiteren bietet ML-Agents einen einfachen Einstieg, da eine Menge an Beispielszenen zur Verfügung gestellt werden. Durch die Betrachtung verschiedener Problemstellungen in diesen Beispielen, lassen sich Orientierungshilfen für dieses Projekt ableiten. Die Beispiele geben Anhaltspunkte über die Nutzung von Observation (dt.: Beobachtung oder Umgebungswahrnehmung), Decision (dt.: Entscheidungen), Action (dt.: Handlungen), Reward (dt.: Belohnungen) und Trainingskonfiguration. Diese sind, wie im Kapitel… angesprochen, die Hauptaspekte, die das Training des Agenten maßgeblich beeinflussen. Im Folgenden werden die wichtigsten Erkenntnisse aus den Beispielen zusammengefasst. Die Erkenntnisse werden dann im Kapitel… aufgegriffen und auf den hier implementierten Einkaufsroboter angewendet. 

\subsection{Rewards}
\label{rewards}
Nach Betrachtung der Beispielszenen lässt sich für den Bereich der Rewards sagen, dass diese generell spärlich vergeben werden. Dies könnte an der geringen Komplexität der Beispielaufgaben liegen. Ein Projekt betrachtet, das Jonglieren eines Balles auf der oberen Ebene eines Würfels. Der Würfel ist hier der trainierte Agent, welcher sich um sein Zentrum so rotieren muss, dass der Ball nicht von der oberen Ebene herunterrollt. Für jeden Zeitschritt, in dem der Ball auf seinem „Kopf“ bleibt, bekommt der Agent eine kleine Belohnung. Sollte der Ball herunterfallen, bekommt er eine hohe Bestrafung. Die Vergabe von Rewards erfolgt mithilfe von Gleitkommazahlen. In dem eben beschriebenen Beispiel erhält der Agent +0.1 für jeden Zeitschritt, in dem der Ball nicht herunterrollt und -1.0, falls der Ball fällt. Belohnungen und Bestrafungen werden in dem Toolkit einheitlich als Reward bezeichnet. Darum wird das im Folgenden ebenso von mir benutzt. Generell gelten für die Vergabe von Rewards diese Ansätze:
\\
\begin{itemize}
	\item Existenzabzüge, also das Verbleiben in einer ungewollten Situation 
	\item Existenzbelohnung, das Bestehenbleiben in einer erwünschten Situation
	\item Erreichen eines Ziels oder den Abschluss einer Aufgabe, wobei unterschieden werden kann, wie gut die Aufgabe abgeschlossen wurde
	\item Erreichen eines Zwischenziels oder einer Teilaufgabe
	\item Abzüge für das Verlassen eines vorgegebenen Bereichs
	\item Für strukturell komplexere Agenten wie Ragdolls (dt.: Lumpenpuppe):
	\begin{itemize}
		\item Ausrichtung des Agenten stimmt mit dem des Ziels überein
		\item Geschwindigkeit des Agenten gleicht dem des Ziels
	\end{itemize} 	 
\end{itemize}

\subsection{Observation und Decision}
\label{observation}
Bei der Betrachtung der Observation wird zwischen zwei Bereichen unterschieden. Einerseits die sogenannte Vector Observation (dt.: Vektor Beobachtung) und andererseits Visual Observation (dt.: visuelle Beobachtung). Zu den üblichen Vector Observationen gehören:
\\
\begin{itemize}
	\item Position und Rotation des Agenten
	\item Position und möglicherweise die Geschwindigkeit des Ziels
	\item Geordnete Liste mit Positionen mehrerer Ziele
	\item Zusätzlich Winkelgeschwindigkeit von Gliedmaßen bei Ragdoll Agenten
	\item Boolean, ob eine Situation zutrifft. Ein Beispiel hierfür wäre, ob sich der Agent auf dem Boden befindet
\end{itemize}
\noindent
\\
Eine andere Variante der Vector Observation ist die Verwendung von sogenannten „Ray Perception Sensor“ (dt.: Strahlen-Wahrnehmungssensor). Diese sind eine von ML-Agents mitgelieferte Komponente, die man am ehesten mit den in der Realität benutzten Lidar Sensoren vergleichen kann. Ein Lidar Sensor liefert durch das Aussenden von Laserstrahlen Daten über die Entfernung zu Objekten in der Umgebung. Diese Werte können in einer Punktwolke gespeichert werden, die daraufhin für die Visualisierung der Daten benutzt kann. Die „Ray Perception Sensor“ ermöglichen die Erkennung von verschiedenen Objekten. Hierfür werden sogenannte Tags verwendet. Diese kann man Objekten in der Szene hinzufügen. Zum Beispiel kann ein Regal den Tag „Regal“ bekommen. Offensichtlich versteht der Agent nicht, was ein Regal ist, aber andere Informationen können so gelernt werden. Hierzu gehört möglicherweise die Breite eines Regals oder auch die Folgen, wenn der Agent in ein Regal hineinfährt.  
\\
Zu den Visual Observation gehört die Nutzung einer Kamera. Diese hat meistens eine niedrige Auflösung und stellt eine Draufsicht, Seitenansicht oder die Sichtweise von der Perspektive des Agenten dar. 
\\
Die Decision sind verhältnismäßig unkompliziert. Der Agent muss nach einer vorgegebenen Anzahl von Steps (dt.: Zeitschritten) eine Auswahl hinsichtlich seiner Action-Parameter treffen. In Unity beträgt die Dauer eines Steps 0.02 Sekunden. Die vorgegebene Anzahl an Steps wird dann mit den 0.02 Sekunden multipliziert und so erhält man den ML-Agents spezifischen Environment Step (dt.: Umgebungszeitschritt). Genauer ausgedrückt, sammelt der Agent bei jedem Environment Step Informationen über seine Umgebung, leitet diese an seine entscheidungstreffende Policy weiter und erhält dann einen Aktionsvektor zurück. Die Anzahl der Unity Steps können im Agenten eingestellt werden. Der Environment Step wird außerdem für die TensorBoard Visualisierung verwendet.

\subsection{Action}
\label{action}
Die Actions werden in „Continuous Actions“ und „Discreate Branches“ unterschieden. Die „Continuous Actions“ können Werte zwischen -1 und 1 annehmen. Dagegen werden die Werte, die ein „Discreate Branch“ annehmen kann, durch seine Größe beeinflusst. Ein Branch der den Wert 3 besitzt, kann nur den Wert 0,1 oder 2 annehmen. Diese Werte werden dann im Code zu Bewegungs- oder Handlungsaktionen umgewandelt. Beispielsweise könnte hier -1 für volle Geschwindigkeit Rückwärts und 1 für volle Geschwindigkeit vorwärts stehen. Die Werte dazwischen reihen sich dann ein. Somit beschreiben diese Werte den Entscheidungsraum des Agenten. 

\subsection{Trainingskonfiguration}
Die Trainingskonfigurationen stellen einen der wichtigsten Aspekte des Trainingsprozesses dar. Sie bestimmen darüber, welcher Algorithmus zum Training verwendet wird. Des Weiteren können hier Einstellungen zum Aufbau des Gehirns getätigt werden. Wie hoch die Lernrate ist, wie schnell diese mit der Zeit abnimmt und vieles mehr. Die Anpassung der Werte innerhalb der Trainingskonfigurationen können einen massiven Einfluss auf das Ergebnis haben. 
\\
Aber bevor ich auf die wichtigsten Parameter eingehe, zurück zum Anfang, um zu entscheiden welcher Lernalgorithmus verwendet wird. Das Toolkit liefert drei Reinforcement Learning Algorithmen mit sich. Hierzu gehören „Proximal Policy Optimisation“ (Abk.: PPO), Soft Actor-Critic (Abk.: SAC) und MultiAgent POsthumous Credit Assignment (Abk.: MA-POCA). 
\\
Wie schon vom Namen abzuleiten ist, wird MA-POCA für Trainingsumgebungen verwendet, in denen mehrere Agenten kooperativ versuchen eine Aufgabe zu lösen. Da für dieses Projekt, aber die nur die Nutzung von einem Agenten pro Umgebung vorgesehen ist, kann dieser Algorithmus ausgeschlossen werden.\cite{mapoca}
\\
SAC trainiert mithilfe von aufgenommenen Erfahrungen, welche immer wieder abgespielt werden, wodurch dieser Algorithmus effizienter hinsichtlich der Trainingsdaten ist.\cite{sac}
\\
PPO wird eher für allgemeine Aufgaben verwendet und ist zuverlässiger.\cite{mapoca} Wegen der fehlenden Beispieldaten, welche erst den SAC-Algorithmus generiert werden müssten und der höheren Zuverlässigkeit, wird in diesem Projekt der PPO Lernalgorithmus verwendet.
\\

\begin{table}[ht]
\centering
\resizebox{\textwidth}{!}{\begin{tabular}{ |c|c|c|c| }
	\hline
	Parameter & Default-Wert & Wertebereich & Erklärung\\
	\hline
	\texttt{batch\_size} & 64 & 32 - 512 & \makecell{Anzahl der Erfahrungen, die für \\eine Iteration der Aktualisierung des\\ Gradientenabstiegs verwendet wird.}\\
	\hline
	\texttt{buffer\_size} & 10240 & 2048 - 409600 & \makecell{Anzahl an Erfahrungen, die gesammelt \\ werden sollen, bevor die Policy \\ aktualisiert wird.} \\
	\hline
	\texttt{time\_horizon} & 64 & 32 - 2048 & \makecell{Anzahl der Erfahrungen, die gesammelt \\ werden sollen, bevor sie dem Buffer \\(\texttt{buffer\_size}) hinzugefügt werden.}\\
	\hline
	\texttt{learning\_rate} & 3e-4 & 1e-5 - 1e-3 & \makecell{Die Lernrate für den Gradientenabstieg \\ am Anfang des Trainings.}\\
	\hline
	\texttt{hidden\_units} & 128 & 32 - 512  & \makecell{Anzahl der Neuronen pro versteckter \\ Ebene in dem Neuronalen Netz.}\\
	\hline
	\texttt{num\_layer} & 2 & 1 - 3 & \makecell{Anzahl der versteckten Ebenen im \\ Neuronalen Netz.}\\
	\hline
	\texttt{gamma} & 0.99 & 0.8 - 0.995 & \makecell{Beschreibt, wie weit sich der Agent um \\mögliche Belohnungen in die Zukunft \\kümmern soll.}\\
	\hline
	\texttt{max\_steps} & 500000 & 5e5 - 1e7 & \makecell{Anzahl der zu durchlaufenden Steps bis \\der Trainingsprozess beendet wird.}\\
	\hline
	\texttt{keep\_checkpoints} & 5 & / & \makecell{Anzahl von Modellen, die \\ zwischengespeichert werden.}\\
	\hline
	\texttt{checkpoint\_interval} & 500000 & / & \makecell{Anzahl an Erfahrungen, die gesammelt \\werden, bis ein Modell \\zwischengespeichert wird.}\\
	\hline
\end{tabular}}
\linebreak
\caption[Zusammenfassung der wichtigsten Hyperparameter für PPO]{Frei übersetzte Zusammenfassung der wichtigsten Hyperparameter für PPO\cite{config_ppo}}
\label{tab:hyperparameter}
\end{table}

\subsection{Installation und Nutzung}
	\newpage
\section{Generierung der Simulationsumgebung}
Die Generierung der Trainingsumgebung ist ein Hauptaspekt des Projekts. Der Agent soll in einer möglichst realistischen Supermarkt-Simulation lernen einen Einkauf abzuschließen. Hierfür werden Regeln und Annahmen benötigt auf deren Basis die Generierung erfolgt. 
\\
Um die Annahmen besser nachvollziehen zu können, muss kurz erklärt werden, wie die Erstellung der Umgebung ablaufen soll. Grundsätzlich wird das Umfeld mithilfe mehrerer 2D Grids (dt.: Gitternetze) generiert. Diese beinhalten Informationen über die Position der Regale und Hindernisse. Mit diesen Daten werden dann Assets wie modellierte Regale geladen, welche an die gespeicherte Position gesetzt werden. Somit wird eine dreidimensionale Umgebung erstellt. Die Regale sind dabei modulare Elemente. Ziel ist es in jeder Iteration der Erstellung des Supermarkts eine neue zufällige Umgebung zu generieren, so dass der Agent keine Umgebung einfach auswendig lernt. Diese Zufälligkeit ist jedoch nur begrenzt umsetzbar und es kann nicht gewährleistet werden, dass die Umgebung nicht schon einmal so erstellt wurde. Dies stellt kein Problem dar, solange sich das Umfeld zwischen jeder Iteration ausreichend verändert. 
Im Folgenden werden die verwendeten Annahmen, sowie die Visualisierung der Umgebung erklärt. Zusätzlich wird genauer beschrieben, wie das Simulationsumfeld generiert und genutzt wird.

\subsection{Getroffene Annahmen für die Implementierung des Projekts}
\label{annahmen}
Für die Implementierungen dieses Projekts wurden diverse Annahmen getroffen, um die Modellierung dieser Aufgabe zu vereinfachen und überhaupt erst zu ermöglichen. Die Annahmen können hierbei in zwei Bereiche unterteilt werden. Zum einen betreffen diese die Strukturierung des Supermarktes. Zum anderen die Aktionen und die Sensorik des Agenten.
\\
Um die Simulation des Supermarktes in der Wirklichkeit zu verankern, habe ich mir mehrere Supermärkte angeschaut und den hier ansässigen Edeka ausgemessen. Dabei habe ich einige grundsätzliche Informationen zusammentragen können. 
\\
\begin{itemize}
	\item Die Abstände zwischen den Regalen liegen bei ungefähr 2 Metern, hierbei gibt es einige wenige Ausnahmen in Verbindung mit Aktionsaufstellern, diese werden aber außenvorgelassen.
	\item Strukturell beginnen die meisten Supermärkte mit frischen Produkten wie Obst, Gemüse, Brot, usw. Danach folgen häufig Kühlbereiche mit Fertigprodukten. Daraufhin länger haltbare Nahrungsmittel wie Süßigkeiten, Dosen, orientalische Produkte, usw. Zum Schluss folgen oft Getränke.
	\item Es gibt keine Sackgassen. Der Weg in das Abteil ist nie der einzige weg hinaus.
	\item Meist wird versucht ein möglichst hohes Maß an Abdeckung mit Regalen zu erreichen.
\end{itemize}
\noindent
\\
Auch wenn dies im Realen nicht immer gilt, wird für die Simulation angenommen, dass der Grundschnitt des Supermarktes sowie die Aufteilungen der Abteile quadratisch oder rechteckig sind. Die Grundfläche wird aus dem zweiten Stichpunkt abgeleitet, in vier Abteile aufgeteilt. Dazu gehören Eingang, Obst und Gemüse Bereich, Abteil mit länger haltbare Güter und Getränke Bereich. 
\\
Des Weiteren wurde für die Berechnung der Wegplanung angenommen, dass der Supermarkt mit der derzeitigen Regalstruktur kartografiert wurde. Zu dieser Karte gehören nur statische Elemente, die sich über die Zeit hinweg nicht ändern. Hindernisse wie Einkaufswägen und Pappboxen mit neuen Artikeln werden nicht miteinbezogen. Sollte sich die Regalanordnung ändern, muss auch die Karte erneuert werden. Ein Ansatz mit einer dynamischen Karte, in der Hindernisse während des Trainingsprozesses vom Agenten hinzugefügt oder gelöscht werden, wird in dieser Arbeit nicht betrachtet. Grundlage hierfür ist, einerseits die Rechenersparnis hinsichtlich der festgelegten Wegplanung. Andererseits ist die Betrachtung der Reaktion des Agenten auf unbekannte Hindernisse ein wichtiger Bestandteil der Arbeit. Das Kartografieren des Supermarktes ist simpel und realistisch umsetzbar. Jedoch gilt dies nicht für die Position der Artikel innerhalb des Supermarktes. Hierfür wird angenommen, dass eine Datenbank besteht, die diese Position eingespeichert hat und innerhalb der Karte darstellen kann. Diese Annahme ist eher unrealistisch, da die Datenbank ständig angepasst werden müsste, sobald die Produkte innerhalb der Regale umsortiert werden, oder neue Artikel hinzukommen. Des Weiteren ist dies besonders fehleranfällig, da der Agent davon ausgeht, dass der ihm übergebene Weg, der Richtige ist. Falls aber die Datenbank nach dem Umräumen nicht vollständig angepasst wurde, würde der Agent ein falsches Produkt einsammeln. Dennoch wurde diese Annahme getroffen, um in der vorgegebenen Zeit dieses Projekt umsetzen zu können. Eine Alternative zu diesem Ansatz, wäre die Entwicklung einer Künstlichen Intelligenz, welche die Produkte mithilfe einer Kamera erkennen kann. Ohne die potenziellen Probleme dieser Vorgehensweise zu beschreiben, wäre hierfür zunächst ein Datensatz mit tausenden Bildern der jeweiligen Produkte aus verschiedenen Perspektiven notwendig. Die Implementierung einer solchen KI als Instanz vor der eigentlichen Untersuchung übersteigt den Rahmen einer Masterarbeit. 
\\
Zusätzlich wird das Einsammeln der Artikel simplifiziert und dabei wird getestet, ob der Roboter zur berechneten Position fährt. Das physische Einsammeln von Objekten verschiedener Größe, während der Roboter nicht unbedingt perfekt ausgerichtet ist, ist hochkomplex, weswegen diese Form der Vereinfachung gewählt wurde.
\\
Dies sind die hauptsächlichen Annahmen, die getroffen wurden. Des Weiteren gibt es zusätzliche Anpassungen, um die Berechnung der Simulation zu optimieren. Für die Kollisionsberechnung der Regale und der Kasse wurde jeweils eine große Hitbox benutzt. Für mehr Realismus hätte man hier die genau Form der Regale durch Hitboxen nachbauen können. 
\\
Des Weiteren wurden keine zufälligen Fehler zu den Sensordaten und auch der Steuerung hinzugefügt. Alle Berechnungen wurden genau übergeben. Grund hierfür war, dass der Agent lange kein sinnvolles Verhalten trainiert hat und das Hinzufügen von Fehlern keinen Mehrwert gebracht hätte.
\\
Bei der Sensorik wurden zusätzlich noch weitere Vereinfachungen gewählt. Für das Projekt wurden keine realistisch funktionierenden Lidar- oder Ultraschallwellensensoren benutzt. Stattdessen wurde die Ray Perception Sensor Komponente verwendet, welche ML-Agents mitliefert. Diese ist für die Nutzung mit Agents ausgelegt und arbeitet ähnlich wie ein Lidar. Darum liegt der Grad der Abstraktion für diese Arbeit im Rahmen. Zusätzlich wird angenommen, dass die Position des Agenten sowie die Position des Ziels genau bekannt ist. Diese Werte sind notwendig, um zu erkennen, ob der Agent das Zier erreicht hat. Dies könnte durch die Nutzung von Kameradaten in einem realen Supermarkt zu einem gewissen Grad umgesetzt werden. Auf die Sensorik sowie die anderen Observationen wird genauer im Kapitel \ref{sensorik_agent} eingegangen.
\\
Hinsichtlich der Visualisierung wäre die Modellierung tausender Artikel extrem zeitaufwendig und würde von einem Trainingsaspekt aus nichts ändern. Deswegen werden die Regale mit Platzhaltermodellen ausgestatte. Diese sehen je nach Bereich unterschiedlich aus, um die Abteile besser visuell unterscheiden zu können. 
\\
Weitere Daten, die ich durch die Ausmessung erhielt, habe ich in die Modellierung des Supermarktes einfließen lassen. Dazu gehört beispielsweise die Höhe der Regale, die Höhe des höchsten Regalfaches, die Abstände zwischen den Fächern und weitere Elemente.
\subsection{Visualisierung}
\label{visualisierung}
Zwar würden für die Implementierung des Versuchsaufbaus die von Unity vorgegebenen Primitive ausreichend sein, um die reine Funktionalität des Ansatzes zu testen. Jedoch wäre dies für den Nutzer wenig anschaulich und kann teilweise, zu Fehlern in der Anwendung führen. Dies kann passieren, wenn nicht genügend visuelle Anhaltspunkte gegeben werden, um die Richtigkeit der Anwendung damit zu überprüfen. Die Betrachtung des Fehlerelements ist aber ein beiläufiger Aspekt und soll deswegen nicht weiter in diesem Kapitel betrachtet werden. 
\\
Um die Visualisierung anschaulicher zu gestalten, müssen verschiedene Modelle erstellt werden, die in ihrem Stil kohärent sind. Dies erhöht die Immersion des Nutzers. Ich habe mich für eine stilisierte Umgebung entschieden, die realistische Elemente mit einem Cartoon Style paart. Ziel war es, einen verspielten Stil zu implementieren, der die Erkennbarkeit der Objekte jedoch nicht erschwert. 
Für die Modellierung wurde die Open Source Software Blender verwendet. In dieser können sowohl 3D wie auch 2D Objekte erstellt werden. Blender liefert eine Vielzahl an Funktionalitäten, von denen für dieses Projekt nur ein Bruchteil verwendet wurde. Modellierung, Sculpting, VFX, Animation, Rigging und vieles mehr, um einmal einige Funktionen aufzuzählen, ermöglicht Blender.\cite{ble} Im Folgenden wird genauer auf den Ablauf bei der Modellierung der Szenenelemente und den Aufbau des Roboters eingegangen. Die Erklärungen umfassen größtenteils Abstrahierungen und sind keine Schritt für Schritt Anleitung zum Nachbauen.

\subsubsection{Visualisierung der Umgebung}
\label{vis_umgebung}
Bei der Modellierung der Umgebungsobjekte gab es zwei wichtige Faktoren. Einerseits die Umsetzung des geplanten Stils und andererseits die Anzahl der “Vertices“ möglichst klein zu halten. Üblicherweise werden, um das Training zu beschleunigen, mehrere Instanzen der Simulationsumgebung erstellt. Eine detailgetreue Modellierung würde dann dazu führen, dass sich deutlich mehr “Vertices“ in der Szene befinden. Diese benötigen bei der Darstellung mehr Rechenleistung und können somit den Trainingsprozess negativ beeinflussen. Beispielsweise könnte eine Folge des höheren Rechenaufwands weniger Instanzen der Szene sein. Dadurch würde die Dauer des Trainings sich um Minuten bis Stunden verzögen. Was wiederum dazu führt, dass weniger verschiedene Ansätze getestet werden können. Somit entsteht ein Dilemma zwischen der Generierung einer visuell ansprechenderen Umgebung und der damit einhergehenden erhöhten Trainingsdauer. Da der Fokus der Arbeit aber auf der Implementierung eines RL-Agenten liegt, ist der visuelle Aspekt eher zweitrangig. Somit muss ein Mittelweg gefunden werden, der eher in Richtung Performance getrimmt ist. 
\\
Mein Ziel für den stilistischen Teil der Modellierung war, dass keine oder nur selten scharfe Kanten zu erkennen sind. Dies lässt sich durch die Funktion "Shade Auto Smooth" in der Blender Version 4.0 und höher umsetzen. In der Abbildung kann man den Unterschied noch einmal genauer erkennen. Um die klar erkennbaren Übergänge zu verhindern, muss der Winkel zwischen den Kanten eines Objektes, eine gewisse Gradzahl unterschreiten. Dafür kann man einerseits den Winkel in den Einstellungen anpassen. Wird der Winkel zu hoch gewählt, sieht die Beleuchtung des Objektes unnatürlich aus. Ist die Einstellung zu gering, sind die Kanten wieder erkennbar. Die andere Option ist die Erhöhung der Anzahl der Kanten, um die Übergänge natürlicher aussehen zu lassen. Dies führt aber wieder zu dem vorher angesprochenen Dilemma. 
\\
Nun gibt es zwei Ansätze für unterschiedliche Objektformen. Bei rechteckigen Objekten lohnt es sich den sogenannten "Bevel Modifier" zu verwenden. Dieser fügt an den äußeren Kanten neue Topologie hinzu, um diese abzurunden. Dadurch reflektiert das einfallende Licht deutlich natürlicher und das modellierte Objekt wirkt realistischer. Dafür reicht es schon wenige Kanten hinzuzufügen, wodurch die Menge an „Vertices“ nur geringfügig steigt. Runde Modelle sind wiederum ein wenig komplizierter. Diese wirken bei genauerer Betrachtung erst rund, wenn die Anzahl an Kanten verhältnismäßig hoch ist. Dies führt automatisch dazu, dass runde Modelle mehr „Vertices“ benötigen als Rechteckige. Hier muss eher beachtet werden aus welcher Entfernung das Objekt betrachtet wird und wie hoch die Auflösung sein sollte, damit die Kanten nicht auffallen. 
\\
Meine Erfahrung in diesem Bereich liefert mir ungefähre Anhaltspunkte, welche Anzahl an „Vertices“ in Abhängigkeit von der Objektform gerechtfertigt sind. Für kleine rechteckige Objekte wie zum Beispiel die Artikel des Supermarktes habe ich mir eine Grenze von 1000 Vertices gesetzt. Runde Artikel hingegen können bis zu 3000 Vertices haben. Die Anzahl ist nicht besonders hoch und befindet sich eher im Low Poly Bereich.

\subsubsection{Visualisierung des Agenten}
\label{vis_agent}
Diese Grenzen gelten jedoch nicht für Objekte, die eher im Fokus liegen wie der Roboter. Hier sind deutlich höhere aufgelöste Topologien in Ordnung, da diese auch nur einmal pro Simulationsumgebung vorkommen. Für die Modellierung des Einkaufsroboters habe ich mir den Prototypen der TU Chemnitz als Anhaltspunkt ausgesucht siehe Abb. Der Roboter ist mit einem Greifarm ausgestattet und besitzt einen Korb in dem die eingesammelten Objekte gelagert werden können. 

\subsection{Generierung und Parameter}
Zusammengefasst für die Generierung. Aus den Erkenntnissen wird für die Simulation abgeleitet, dass der Abstand zwischen den Regalen mindestens 2 Meter betragen muss. Das verhindert automatisch die Generierung von Sackgassen. Außerdem wird versucht, die Abdeckung mit Regalen möglichst hoch zu wählen. 
\subsection{stationäre Hindernisse}

	\newpage
\section{Strukturierung des Agenten}
\subsection{Steuerung}
\subsection{Observation}


	\newpage
\section{Trainingsablauf}
\subsection{Anfänge...}
\subsection{Curriculum Learning}
\subsection{Training im Laufe der Zeit}
\label{training_zeit}
\subsection{Erkenntnisse und Learnings}
Würde ich eher in die Auswertung packen.


	\newpage
\section{Auswertung}


	\newpage
\section{Ausblick}
\label{ausblick}
Ein wichtiger Bestandteil der Betrachtung, die hier wegen fehlender Zeit ausgelassen wurde, ist der Umgang des Agenten mit dynamischen Hindernissen. Einkaufsroboter müssen dazu in der Lage sein mit dem Verhalten von Menschen umzugehen. Diese weisen in einem realen Umfeld unterschiedliche Muster auf. Teilweise würden sie dem Roboter ausweichen, andere stellen sich ihm vielleicht aktiv in den Weg oder bleiben an Ort und Stelle stehen. Herauszufinden wie der Roboter dieser vorm von Hindernissen effizient ausweicht, während er weiterhin versucht, alle Artikel der Einkaufsliste zu kaufen, könnte Bestandteil einer weiterführenden Arbeit sein. 
\\
Ebenso ist die Implementierung eines Vergleichsmodell von Interesse. Dieses könnte beispielsweise ein regelbasiertes System sein, welches gleichbleibend mit Hindernissen umgeht. Für die Betrachtung der verschiedenen Systeme würde man diese in getrennten Umgebungen, die über dieselben Hindernisse und zu kaufende Artikel verfügen, gegeneinander antreten lassen. Durch die Zufälligkeit der Umgebung kann nicht gewährleisten werden, dass die simulierten Menschen nach Kontakt mit dem Roboter gleich agieren. Dennoch würde die Nutzung eines Vergleichsmodells Anhaltspunkte über die Effizienz des KI-Agenten geben. Durch diese Form der Betrachtung könnte man einfacher die Stärken und Schwächen der jeweiligen Ansätze aufschlüsseln. Hierfür sollten Gesichtspunkte wie Dauer des Einkaufs, Anzahl an Kollisionen und Verärgerung der simulierten Menschen analysiert werden. Wenn man den Agenten auf Basis dieser Gesichtspunkte optimiert, wäre das Ergebnis, ein Roboter, der so wenig wie möglich Fahrtweg zurücklegt, während er die anderen Einkäufer minimal stört und die Dauer des Einkaufs gering hält. 
\\
Zusätzlich sollten Vereinfachungen, die in dieser Implantierung vorgenommen wurden, sukzessive abgebaut werden. Des Weiteren könnten noch komplexere Regalstrukturen, wie man sie aus dem Edeka oder Rewe kennt, hinzugefügt werden. Desto weniger Simplifizierungen verwendet werden, umso aussagekräftiger sind die Ergebnisse.

	\newpage
	\printbibliography
\end{document}
